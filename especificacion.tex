\documentclass[spanish, a4paper, 12pt] {article}
\usepackage[spanish]{babel}
\usepackage[utf8x]{inputenc}
\usepackage{amsmath}
\usepackage{amssymb}
\usepackage{amsfonts}
\usepackage{latexsym}
\usepackage{mathtools}
\usepackage{anysize}
%\marginsize{2cm}{2cm}{2cm}{3cm}
\newcommand\eqdef{\stackrel{\mathclap{\mbox{\tiny{def}}}}{=}}
\newcommand\eqac{\stackrel{\mathclap{\mbox{*}}}{=}}

\usepackage{graphicx}
\usepackage{hyperref}
\usepackage{float}
\usepackage{verbatim}
\usepackage{textcomp}
\DeclareGraphicsExtensions{.pdf,.png,.jpg}

\usepackage[a4paper,bindingoffset=0.2in,left=0.8in,right=0.8in,top=1.1in,bottom=1in,footskip=.25in]{geometry}

\newcommand{\lname}[0]{CeMAGR+ }

\begin{document}
\title{\lname: primer boceto}
\author{Marco Antonio Garrido y Celia de Frutos}
\date{}
\maketitle
Nuestra intención para este lenguaje de programación es escoger lo mejor y lo peor de cada lenguaje que conocemos y juntarlo todo en un único lenguaje para acabar con las discusión de oficina en las que uno se burlan de otros por ser de C++ y no de Java.
\subsection*{Declaración de variables y definición de funciones}
Empecemos por lo más básico:
\begin{verbatim}
<declaración-variable> ::= <tipo> <declarador>
<tipo> ::= bool | int
<declarador> ::= €<identificador> | <declarador>(<expresión-constante-arit>)
\end{verbatim}
\lname cuenta con dos tipos: booleanos y enteros. Los nombres de variable van precedidos por el símbolo del euro (PHP nos inspira) y se permite la declaración de arrays multidimensionales declarando la dimensión entre paréntesis (MATLAB nos inspira pero menos).\\

\lname es un lenguaje sin orientación a objetos pero que permite el uso de funciones funciones.
\begin{verbatim}
<def-función> ::= func <tipo> <identificador> (<variables-entrada>) {<bloque-func>}
<variables-entrada> ::= <declaración-variable>
                        | <variables-entrada>, <declaración-variable>
\end{verbatim}
donde bloque-func será un bloque de instrucciones.
\subsection*{Instrucciones}
Las instrucciones deben de usarse dentro de una función y el punto de entrada para nuestro programa es la función main.\\

Todo bloque de instrucciones de una función tiene que terminar con la instrucción return.
\begin{verbatim}
<bloque-func> ::= <bloque> <return>
<bloque> ::= <instrucción> | <bloque> <instrucción>
<return> ::= return <expresión>;
<instrucción> ::= <declaración-variable>; | <bloque-if> | <bloque-loop>
                  | <bloque-foop> | <asignación> | <llamada-func>
\end{verbatim}
Hablemos en primer lugar de la asignación.
\begin{verbatim}
<asignación> ::= <declarador> := <expresion>
<expresion> ::= <exp-arit> | <exp-bool>
<exp-arit> ::= <expresión-constante-arit> | <declarador> | <llamada-func>
               | <exp-arit> + <exp-arit> | <exp-arit> - <exp-arit>
               | <exp-arit> * <exp-arit> | <exp-arit> / <exp-arit>
               | <exp-arit> % <exp-arit>
<exp-bool> ::= <expresión-constante-bool> | <declarador>
               | <exp-bool> & <exp-bool> | <exp-bool> '|' <exp-bool>
               | not <exp-bool> | <exp-arit> == <exp-arit>
               | <exp-arit> != <exp-arit> | <exp-arit> < <exp-arit>
               | <exp-arit> > <exp-arit> | <exp-arit> <= <exp-arit>
               | <exp-arit> >= <exp-arit>
\end{verbatim}
Solo existe un operador de asignación. Nadie necesita los `+=' de C.\\

Las llamadas a función son como en C.
\begin{verbatim}
<llamada-func> ::= <identificador>(<variables>)
<variables> ::= <declarador> | <variables>, <declarador>
\end{verbatim}
Por último, hablemos de las estructuras de control de ejecución.
\begin{verbatim}
<bloque-if> ::= If (<exp-bool>) {<bloque>}
              | If (<exp-bool>) {<bloque>} Else {<bloque>}
<bloque-loop> ::= Loop (<exp-bool>) {<bloque>}
<bloque-foop> ::= Foop (<asignación>; <exp-bool>; <asignación>) {<bloque>}
\end{verbatim}
\subsection*{Ejemplos}
\subsubsection*{Fibonacci}
\begin{verbatim}
func int fib(int n) {
    int ret;
    int prev;
    ret := 1;
    prev := 0;
    int i;
    Foop (i := 0; i < n; i := i + 1) {
        int auxi;
        auxi := ret;
        ret := ret + prev;
        prev := auxi;
    }
    return ret;
}
\end{verbatim}
\subsubsection*{Potencia (recursivo)}
Vamos a calcular la potencia p-esima de n en tiempo logarítmico:
\begin{verbatim}
func int pot(int n, int p) {
    int ret;
    If (p == 0) {
        ret = 1;
    } Else {
        int np;
        np := p/2;
        ret := pot(n, np);
        ret := ret * ret;
        If (p % 2 == 1) {
            ret := ret * n;
        }
    }
    return ret;
}
\end{verbatim}
\end{document}
